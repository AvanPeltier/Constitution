% BEFORE CHANGES ARE MADE TO THIS DOCUMENT:
% -References will be automatically updated if any part is added, deleted, etc. 
%  However, if a sub part is moved to a different part, its references must be 
%  changed.
% -This document must be ratified by the House (as per the Constitution), 
%  then printed, signed, notarized, and placed in the House filing cabinet
%  if changes are to be officialized.

\documentclass{article}
% The xr package allows external references
\usepackage{xr-hyper}
\usepackage{hyperref}

% This package is useful for debugging label problems 
% Comment out in final revision
%\usepackage{showkeys}

% Define the external document to be bylaws for cross referencing purposes
%\externaldocument{bylaws}[https://github.com/ComputerScienceHouse/Constitution/blob/master/bylaws.pdf?raw=true]
\externaldocument{bylaws}[https://github.com/ComputerScienceHouse/Constitution/blob/master/bylaws.pdf?raw=true]

% Title page information
\title{Computer Science House Constitution}
\author{Computer Science House Constitution Committee}
% Last Modified Date
\newcommand{\datechanged}{Proposed: \today}
\date{\datechanged}

% Fix margins
\setlength{\evensidemargin}{0in}
\setlength{\oddsidemargin}{0in}
\setlength{\textwidth}{6.5in}
\setlength{\topmargin}{0in}
\setlength{\textheight}{8.5in}

% Use \article for articles and \asection for sections of articles.
% Automatically provide labels with the same article or section title.
\newcommand{\article}[1]{\section{#1} \label{#1}}
\newcommand{\asection}[1]{\subsection{#1} \label{#1}}
\newcommand{\asubsection}[1]{\subsubsection{#1} \label{#1}}
\renewcommand{\thesection}{Article \Roman{section}}
\renewcommand{\thesubsection}{Section \arabic{section}.\Alph{subsection}}
\renewcommand{\thesubsubsection}{\arabic{section}.\Alph{subsection}.\arabic{subsubsection}}

% Adding an \asubsubsection -- I feel dirty
\setcounter{secnumdepth}{5}
\newcommand{\asubsubsection}[1]{\paragraph{#1} \label{#1}}
\renewcommand{\theparagraph}{\arabic{section}.\Alph{subsection}.\arabic{subsubsection}.\Alph{paragraph}}

% Headings
\pagestyle{myheadings}
\markright{{\rm CSH Constitution \hfill \datechanged \hfill Page }}

% Reference example:
%Test reference \ref{House Objectives} House Objectives.


\begin{document}
% Title
\maketitle

% ARTICLE I - INTRODUCTION
\article{Introduction}

\asection{Name}
The name of this Special Interest House shall be Computer Science House.

\asection{Derivation of Authority}
The Computer Science House shall recognize that it receives its right to function as a Special Interest House from the Department of Residence Life in conjunction with the Residence Halls Association.

\asection{House Objectives}
The objectives of Computer Science House are:
\begin{enumerate}
	\item To enhance the education experience of its members.
	\item To offer students educational programming with an emphasis in computers.
	\item To provide a variety of services for its members, the RIT campus, and the Rochester Community.
	\item To provide a friendly and comfortable living environment in the residence halls.
\end{enumerate}

% ARTICLE II - CONSTITUTIONAL STRUCTURE
\article{Constitutional Structure}

\asection{House Charter}
The House Charter is a document drafted by the department of Residence Life, setting down the guidelines within which this House operates and from which it derives its authority.

\asection{Constitution}
The House Constitution is written and maintained by the House and defines the major aspects, goals, and governing structure of the house. It is reviewed annually by the Department of Residence Life. The House Constitution is comprised of two sections: articles and by-laws. Articles define the major aspects of the House. A by-law may expand on rules outlined in an article, but they must never contradict a Constitutional article.

\asubsection{ Non-Semantic Changes}
Any non-semantic change to the Constitution should be presented to the House for discussion. After the discussion an immediate House vote is taken.A quorum of fifty percent of the Total Number of Possible Votes must be present in order for the vote to be official. A vote exceeding fifty percent of all House members casting a vote is required for the non-semantic change to be placed into the constitution.

\asubsection{Semantic Changes}
Any semantic change to the Constitution requires the change to be proposed in writing for discussion at a House Meeting. Any modifications made due to the discussion are added to the written proposal and the modified proposal is posted in the House during the week. The final proposal is presented the following week and ballots are distributed for a ballot House vote as described in \ref{Balloted Vote}. The ballots are collected for a minimum of a forty-eight hour period. A quorum of two-thirds
of the Total Number of Possible Votes must cast ballots for the vote to be official. A vote equaling or exceeding two-thirds of the number of votes cast is required for the change to be placed into the constitution. The Constitution may be overridden by an immediate House vote as described in \ref{Immediate Vote}. There must be a quorum of eighty-five percent of the Total Number of Possible Votes for the vote to be official. A vote equaling or exceeding ninety percent of the number of votes cast is required for the override to take effect.

% ARTICLE III - MEMBERSHIP
\article{Membership}
There are five major types of membership available to Computer Science House. Each carries different qualifications, expectations, and privileges. When describing the different memberships available, the following terms are used:
\begin{description}
	\item[Qualifications:] What qualifications an applicant needs to apply.
	\item[Selection:] The process by which an applicant gains membership.
	\item[Expectations:] The duties and responsibilities of House members.
	\item[Privileges:] The benefits offered to House members.
	\item[Evaluations:] The process by which a member is reviewed and assessed.
	\item[Resignations:] The process by which a member terminates House membership.
	\item[Term:] The length of time the membership lasts.
\end{description}

\asection{Introductory Membership}
\asubsection{Introductory Membership Qualifications}
Introductory Membership is open to all students at the Rochester Institute of Technology.
\asubsection{Introductory Membership Selection}
Applicants notify the House of their interest in membership by submitting an application to the Evaluations Director. They must then undergo the selection process as defined in \ref{Selection Processes}.
\asubsection{Introductory Membership Expectations}
Introductory members are expected to meet all the requirements of the Introductory Process, as described in \ref{Expectations of an Introductory Member}.
\asubsection{Introductory Membership Privileges}
Introductory members receive the right to use Computer Science House facilities, to attend House functions, and to have housing priority over all persons who are not Computer Science House members.
\asubsection{Introductory Membership Evaluations}
Introductory members are evaluated on their performance during the introductory period. The introductory evaluation process is described in \ref{Introductory Evaluation}.
\asubsection{Introductory Membership Resignations}
Introductory members may resign by submitting the request for termination of membership to the Chairman, or Evaluations Director in writing before the completion of the introductory process.
\asubsection{Introductory Membership Term}
Introductory membership shall last until the end of the introductory process, at which time active membership is granted or membership is revoked.

\asection{Active Membership}
\asubsection{Active Membership Qualifications}
Active Membership is open to all students currently enrolled at the Rochester Institute of Technology who have passed the Introductory Evaluation and their most recent Membership Evaluation.
\asubsection{Active Membership Selection}
Qualifying members may self-select into Active Membership by paying dues to the Financial Director and notifying the Evaluations Director.
\\*\\*
Any Alumni Member in bad standing (\ref{Alumni Membership Selection}) may become an Active Member by notifying the Evaluations Director of their intent to participate, who will bring them up for vote at the next Evaluations meeting. If a majority is in favor of the return, the Alumni is reinstated as an Active Member with Off-Floor status effective immediately. The decision will be announced at the following House meeting.
\asubsection{Active Membership Expectations}
Active members are expected to be active participants in the House as defined in \ref{Expectations of House Members}.
\asubsection{Active Membership Privileges}
Active Members receive the right to:
\begin{itemize}
	\item Vote on all issues brought before the House
	\item Reside on the House
	\item Hold an Executive Board office on the House
	\item Use Computer Science House facilities
	\item Attend Computer Science House functions
	\item Receive priority for available housing on the House when returning from cooperative education
	\item Guaranteed housing in compliance with the Residence Life policies regarding Special Interest Houses and the Housing Selection Process (if they have On-Floor status)
\end{itemize}
Active Members currently on co-op forfeit their right to vote on house issues, and likewise do not count towards quorum. Exceptions may be made at the discretion of the Evaluations Director.
\asubsection{Active Membership Evaluations}
Active Members are evaluated semi-annually through the Evaluation Process as described in \ref{Evaluations Processes}.  Failure of the Evaluation Process could result in the member being asked to find alternative housing as soon as possible in accordance with all applicable Residence Life policies regarding room changes.
\asubsection{Active Membership Resignations}
An Active member may resign by submitting, in writing, the reason for resignation to the Chairman, or Evaluations Director. Instead of forfeiting membership, Active members who resign may elect to become Alumni defined in \ref{Alumni Membership Selection}. The resignation will take effect immediately and an announcement will be made at the following House Meeting.
\asubsection{Active Membership Term}
Active Membership shall last until the member: resigns, fails the Evaluation Process, or changes membership status.

\asection{Alumni Membership}
\asubsection{Alumni Membership Qualifications}
Alumni Membership is open to all former Active members of Computer Science House who passed at least one Membership Evaluation and departed for reasons other than revocation of House Membership.
\asubsection{Alumni Membership Selection}
Active members who depart house (i.e. resign) after passing the current operating session's Membership Evaluations are considered to be Alumni in good standing.
\\*\\*
Active members who depart house without passing the current operating session's Membership Evaluations are considered to be Alumni in bad standing. This may be appealed to the Executive Board in order to pursue a different outcome.
\asubsection{Alumni Membership Expectations}
There are no expectations associated with the Alumni Membership status.
\asubsection{Alumni Membership Privileges}
Alumni Members shall receive the right to:
\begin{itemize}
	\item Use Computer Science House facilities
	\item Attend Computer Science House functions
\end{itemize}
\asubsection{Alumni Membership Evaluations}
Alumni Members are not subject to any evaluation.
\asubsection{Alumni Membership Resignations}
There are no resignations associated with Alumni Membership status.
\asubsection{Alumni Membership Term}
Alumni Membership shall last indefinitely or until the member chooses to pursue Active Membership.

\asection{Honorary Membership}
\asubsection{Honorary Membership Qualifications}
Honorary Membership is open to a person whom the House feels has contributed great personal effort to the House and is deserving of House recognition.
\asubsection{Honorary Membership Selection}
Any House member may nominate a candidate for Honorary Membership by submitting the name in writing to the Evaluations Director. The Evaluation Director then begins the Honorary Membership Selection Process as defined in \ref{Selection Process for Honorary Members}
\asubsection{Honorary Membership Expectations}
Honorary Members are encouraged to remain in contact with the House.
\asubsection{Honorary Membership Privileges}
Honorary Members may advise in all issues brought before the House, use Computer Science House facilities, and attend Computer Science House functions.
\asubsection{Honorary Membership Evaluations}
Honorary Members are not subject to formal evaluations.
\asubsection{Honorary Membership Resignations}
An Honorary Member may resign by submitting in writing the reason for resignation to the Chairman. The resignation will be read at the following House Meeting and become effective at that time.
\asubsection{Honorary Membership Term}
Honorary Membership shall last until the member resigns.

\asection{Advisory Membership}
\asubsection{Advisory Membership Qualifications}
Advisory Membership is open to all members of the Rochester Institute of Technology professional, academic, or administrative staff.
\asubsection{Advisory Membership Selection}
The Executive Board may open nominations for Advisory Members. The candidates then participate in the Advisory Selection Process as defined in \ref{Selection Process for Advisory Members}
\asubsection{Advisory Membership Expectations}
Advisors are encouraged to offer advice and assistance to the House in any capacity the Advisor is able to help. Advisors are also encouraged to meet the House members and occasionally attend House activities.
\asubsection{Advisory Membership Privileges}
Advisory Members may advise in all issues brought before the House, use Computer Science House facilities, and attend Computer Science House functions.
\asubsection{Advisory Membership Evaluations}
Advisors are not subject to formal evaluations.
\asubsection{Advisory Membership Resignations}
An Advisor may resign by submitting in writing the reason for resignation to the Chairman. The resignation will be read at the following House Meeting and become effective at that time.
\asubsection{Advisory Membership Term}
Advisory Membership shall last until the member resigns.

% ARTICLE IV - EXECUTIVE BOARD
\article{Executive Board}
The Executive Board is the main governing body of the House. Its purpose is to provide leadership and direction for the House, to oversee the day-to-day operations of the House, and to initiate and organize programs and projects for the House. It is composed of the seven permanent directors and the Chairman.
\\*\\*
There is one permanent directorship for each major aspect of the government of the House and each one is chaired by an Executive Board member. Ad-hoc directorships are created on an as-needed basis. They are generally very task oriented and are chaired by a House member. A directorship has some jurisdiction in its area of interest and often is responsible for the day-to-day decisions regarding its area of interest. Any large expenditures or large effect decisions must be brought before the entire House
\asection{Members of the Executive Board}
\asubsection{Chairman}
\asubsection{Voting Members}
\begin{itemize}
	\item Evaluations Director
	\item Social Director(s)
	\item Financial Director
	\item Research and Development Director
	\item House Improvements Director
	\item Operational Communications Director
	\item House History Director
\end{itemize}
\asubsection{Non-Voting Members}
\begin{itemize}
	\item Ad Hoc Directors
\asubsection{House Secretary}
\end{itemize}
\asection{Closed Executive Board}
Closed Executive Board Meetings are open only to the Chairman, Voting Members of the Executive Board, and those with the express permission of the Executive Board. A closed Executive Board meeting may be called at any time by any member of the Executive Board. However, the Chairman and at least two-thirds of the Voting Members must be present for the meeting to be called.

\asection{Responsibilities}
\renewcommand{\theenumi}{\alph{enumi}} % For this section, we want items to use letters
\asubsection{Responsibilities of the Executive Board}
\begin{enumerate}
	\item To hold a weekly meeting specific to their responsibilities and submit notes to the House.
	\item To meet, as an Executive Board, at least weekly during the standard operating session, as defined in \ref{Standard Operating Session} to discuss and report the operations of the House.
	\item To report pertinent information to House members at the following House Meeting.
	\item To maintain records of the goals defined by each previous Executive Board.
	\item To act as a Judicial Board as defined in \ref{Judicial}.
	\item To review major projects, as defined in \ref{Expectations of House Members},  presented to them by the Evaluations director.
	\item To make the final vote regarding conditionals and appeals as defined in \ref{Membership Evaluation}.
	\item To publish a document at the end of each semester to all Members stating House’s accomplishments of that semester.
	\item To review and update the Constitution at the end of each standard operating session, as defined in \ref{Standard Operating Session}.The constitution should remain up to date with current practices.
\end{enumerate}

\asubsection{Responsibilities of the Chairman}
\begin{enumerate}
	\item To preside over Executive Board and House Meetings.
	\item To exercise general supervision over the operations of the Executive Board.
	\item To exercise general supervision over regular House activities.
	\item To act as a liaison to the academic and administrative departments at RIT.
	\item To act as a part of a Judicial Board as defined in \ref{Judicial}.
	\item To cast tie-breaking vote in a split decision in an Executive Board vote.
\end{enumerate}

\asubsection{Responsibilities of the Evaluations Director}
\begin{enumerate}
	\item To preside over Evaluations Meetings.
	\item To exercise general supervision over Evaluations operations.
	\item To oversee the screening, interviewing, and acceptance or rejection of prospective House members.
	\item To oversee the Semi-Annual Evaluations of current House members.
	\item To collaborate with the Residence Life Advisor to determine room change selection and any changes of membership residency.
	\item To act as a part of a Judicial Board as defined in \ref{Judicial}.
	\item To prepare the House for Open Houses, tours, and special events.
\end{enumerate}

\asubsection{Responsibilities of the Social Director}
\begin{enumerate}
	\item To preside over Social Meetings. House members are encouraged to bring ideas for social events to the meetings for discussion.
	\item To exercise general supervision over Social operations.
	\item To oversee the organization, initiation, and execution of House activities and events.
	\item To ensure that there is a variety of activities for House members to participate in throughout the academic year.
\end{enumerate}

\asubsection{Responsibilities of the Financial Director}
\begin{enumerate}
	\item To preside over Financial Meetings. House members are encouraged to bring ideas for fund-raising activities to the meetings for discussion.
	\item To supervise financial administrators and transactions involving house projects.
	\item To maintain financial and inventory records of House capital and assets.
	\item To plan and enforce a House budget.
	\item To oversee House finances and generation of House funds.
	\item To publish a semesterly House financial statement.
	\item To supervise the collection of semesterly House dues.
	\item To oversee the planning and execution of fundraising events.
\end{enumerate}

\asubsection{Responsibilities of the Research and Development Director}
\begin{enumerate}
	\item To preside over Research and Development Meetings. House members are encouraged to bring ideas for projects to the meetings for discussion.
	\item To exercise general supervision over Research and Development operations.
	\item To oversee the planning, organization and construction of technical projects for the House's benefit.
	\item To collaborate with the Operational Communications Director in an attempt to fulfill the House's need for technical equipment.
	\item To provide seminars and tutorials to educate House members in technical areas.
\end{enumerate}

\asubsection{Responsibilities of the House Improvements Director}
\begin{enumerate}
	\item To preside over House Improvements Meetings.
	\item To exercise general supervision over the House Improvements operations.
	\item To oversee the organization and construction of physical improvements to the House.
	\item To oversee the general maintenance of the appearance of the House.
\end{enumerate}

\asubsection{Responsibilities of the Operational Communications Director}
\begin{enumerate}
	\item To represent the Root Type Persons at Executive Board Meetings and at House Meetings.
	\item To report the status of the House computer systems and network to the Executive Board and House members.
\end{enumerate}

\asubsection{Responsibilities of the House History Director}
\begin{enumerate}
	\item To preside over House History Meetings.
	\item To exercise general supervision over the House History operations.
	\item To maintain, uphold, and promote house traditions.
	\item To collaborate with the Social Director(s) to ensure that all the Active, Alumni, and Advisory Members are informed of upcoming events.
	\item To oversee the production and distribution of a semi-annual newsletter.
	\item To oversee the creation of a yearbook outlining House events for the year.
\end{enumerate}

\asubsection{Responsibilities of the House Secretary}
\begin{enumerate}
	\item To ensure that minutes are recorded and posted for Executive Board Meetings.
	\item To ensure that minutes are recorded and posted for House Meetings.
	\item To oversee the maintenance of Executive Board records and documents.
	\item To provide administrative assistance to the Executive Board.
	\item To oversee the maintenance of the Activities Information Centers.
	\item To record all votes cast during House Meetings and Open Executive Board Meetings.
\end{enumerate}

\asubsection{Responsibilities of Ad Hoc Directorships}
\begin{enumerate}
	\item Ad-Hoc Directorships are responsible for the task for which they were created.
	\item Any responsibilities budgets or finances for the Ad-Hoc Directorship are placed upon the assigned director.
\end{enumerate}

\asection{Qualifications}
\asubsection{Qualifications to be the Chairman, Evaluations Director, Social Director(s), Financial Director, Research and Development Director(s), House Improvements Director, House History Director}
\begin{enumerate}
	\item Candidates must be Active Members during the term of office.
	\item Candidates must have at least one full semester of House membership as an Active Member.
	\item Elected or selected candidates may not hold two simultaneous voting Executive Board positions, and must therefore resign their current position or decline a second position should they be elected or selected to a second voting position.
\end{enumerate}
\asubsection{Qualifications to be the Operational Communications Director}
\begin{enumerate}
	\item Candidates must be a current Root Type Person \ref{Operations of the Operational Communications Directorship}
	\item Candidates must be a current Active Member.
	\item A Root Type Person cannot be the director if they currently hold any other Executive Board Position.
\end{enumerate}

\asubsection{Qualifications to be the House Secretary or an Ad Hoc Director}
\begin{enumerate}
\item
 Candidates must be Active Members.
\end{enumerate}

\asection{Selection}
\asubsection{Dual Directorship}
Social and Research and Development are the only voting Executive Board positions that allow for Dual Directorship. If two candidates elect to run as a Dual Directorship, their names are placed together on a single line of the election ballot. If they are also nominated as a normal directorship or as a dual directorship with another House member, their votes are not cumulative. Each different nomination must be a separate entry on the election ballot. 
Non-voting Executive Board positions (i.e. Ad-Hoc Directors) are not restricted to single or dual directorship.

The following special cases cover the operation of a dual directorship:
\begin{itemize}
	\item If one of the members in a dual directorship resigns the directorship, or for any other reason ends the term of office, the other member in the dual directorship must also step down and the office becomes vacated. The vacated office is then handled like any other vacated office; see \ref{Resignations}.
	\item During an official Executive Board Vote, each dual directorship member's vote counts for one-half vote in the tallying of votes. The members of the dual directorship need not vote the same way in a vote.
	\item A member of a dual directorship may not hold any other Executive Board position.
	\item When attendance requirements call for the dual directorship position to be present, both dual directorship members are to fulfill the requirements.
\end{itemize}
\asubsection{Selection of the Chairman, Evaluations Director, Social Director(s), Financial Director, House Improvements Director, House History Director, Research and Development Director(s)}
\begin{enumerate}
	\item The opening of the Executive Board position(s) is announced at a House meeting and nominations for the position are taken for a minimum of a seventy-two hour period. Any House member may nominate any eligible member or pair of members where appropriate for a Directorship.
	\item The candidates meeting the qualifications of the office they were nominated for, will be notified of their nomination. Each candidate is given a minimum of twenty-four hour period to accept or decline the nomination. A list of all nominees who have accepted their nominations will be posted shortly thereafter.
	\item Each candidate will be given an equal amount of time before the House to present their platform of candidacy.
	\item Ballots will then be distributed and voting will be open for a minimum of a forty-eight hour period. The Ballots will list, in random order, all of the candidates for a given office with a means to indicate the selection of one of the candidates. In addition, an area will be provided to indicate a write-in selection of a candidate.
	\item At the end of the voting period, the Chairman will terminate voting by collecting the ballot box and the votes shall be counted as stated in \ref{Balloted Vote}.
	\item The winners are determined by a simple majority vote of the ballots cast as defined in 5.C.1, Voting. A quorum of fifty-percent of the number of all members eligible to vote is required for the election to be official. All winners are notified of their election. If the position is currently vacant, the winner immediately assumes office. If not, the winner will assume office at the end of the current term (\ref{Term}). Any office whose winner declines the election, or whose winner was a write-in candidate who does not fulfill the requirements of the elected office, shall remain vacant and a new election process shall begin for that office.
\end{enumerate}
\asubsection{Selection of the Operational Communications Director}
\begin{enumerate}
	\item A candidate for the Operational Communications Directorship shall be chosen by the current Root Type Persons.
	\item The candidate is given a minimum of twenty-four hours to accept or decline the nomination.
	\item The candidate is presented to the Executive Board for approval. This Executive Board meeting is closed to the Executive Board Members, Root Type Persons, and House members with explicit invitation from the Executive Board.
	\item All current Executive Board Voting Members must be present during the discussion and voting period unless that member is a candidate for the position, in which case the member is absent and their vote is abstained. A simple majority Executive Board vote, as described in \ref{Simple Majority}, is taken to determine whether the candidate is selected for the position.
\end{enumerate}
\asubsection{Selection of Ad Hoc Directors}
\begin{enumerate}
	\item When the Executive Board or a group of House members feels an Ad-hoc Directorship is necessary, they present their plans to the Executive Board and the Executive Board decides by simple majority vote whether directorship status is granted.
	\item When the Ad-Hoc Directorship is granted status, a director is appointed, and duties, budgetary, and membership considerations are defined.
\end{enumerate}
\asubsection{Selection of the House Secretary}
The Executive Board may choose to select a current voting Executive Board member, or to select any interested Active member, as House Secretary. The selection process can be an informal appointment, or follow an election process similar to other voting Executive Board positions.

\asection{Resignations}
An Executive Board Member may resign by submitting in writing the reason for resignation to the House Chairman. The resignation will be read at the following House Meeting and become effective at that time. The office will become vacant and the selection process for a new member will begin at that time. The duties and responsibilities of a vacated office are assumed by the Chairman or by an Active member that is appointed at the Chairman’s discretion until the new member takes office. The vote of a vacated Executive Board position shall be cast as an abstention in all Executive Board matters where this vote is required to be cast.

\asection{Impeachment}
\begin{enumerate}
	\item Impeachment of any Executive Board Member may be initiated by petition, in writing, consisting of a minimum number of signatures of current House members equaling or exceeding one-third the Number of Possible Votes as defined in \ref{Total Number of Possible Votes}.
	\item The impeachment petition is then presented at an Executive Board Meeting. The member(s) initiating the petition present their case to the Executive Board. The Executive Board then questions the accused member of the allegations.
	\item The Executive Board votes on the impeachment petition, with all voting members present except the accused member who must be absent and whose vote counts as an abstention, to determine if the allegations stated in the petition are legitimate grounds for impeachment. If the majority of Executive Board votes are negative, the petition and impeachment proceedings are dismissed. This vote may be overridden by an impeachment petition consisting of the grounds for impeachment and a minimum of number of signatures of current House members equaling or exceeding two-thirds the number of all House members currently eligible to vote.
	\item If the majority of the Executive Board votes are positive, or the negative vote was overridden, the petition is presented at the following House Meeting and the accused and accuser(s) again present their cases.
	\item Ballots will then be distributed and a secret ballot House vote shall be held for a minimum of a forty-eight hour period to determine whether or not the member should be removed from office. Votes shall be collected and counted as described in \ref{Balloted Vote}.
	\item A quorum of two-thirds of the Total Number of Possible Votes is necessary for the vote to be official. A vote equaling or exceeding two-thirds the number of all ballots cast is necessary for the accused officer to be removed from office. If a quorum cannot be reached after two attempts, or the percentage of affirmative votes does not equal or exceed the minimum, impeachment proceedings are dismissed.
	\item If the percentage of affirmative votes equals or exceeds the minimum the impeached officer is relinquished of their position and any benefits thereof and this position is treated like any other vacated position. A new selection and interim duty fulfillment procedure is followed similar to that of a resignation; see \ref{Resignations}.
	\item House Secretary can be impeached according to the above process, or may be dismissed by a Simple Majority vote of the Executive Board.
\end{enumerate}

\asection{Term}
\begin{enumerate}
\item The Election Process for the following year's Executive Board members shall begin in the middle of Spring semester.
\item The newly selected officers shall begin their terms on June 1 of that year and their terms shall end on May 31 of the following year.
\item The term of an officer will be abbreviated due to resignation, impeachment, failure of the evaluation process, or change in membership status.
\item Officers elected or selected during the course of the year due to an abbreviated term of the previous officer shall hold office until the end of the normal term.
\item When the task of an Ad Hoc Directorship has been completed, the directorship dissolves. When an Ad Hoc Director resigns, the directorship dissolves and must be reinstated with a new director.
\end{enumerate}

\asection{Appeals}
Decisions made by the Executive Board may be appealed and overturned. To initiate an appeal, a member must have the support of three voting members of the Executive Board, or a petition with the signatures of one-third of Active Members. After the appeal is presented, a Simple Majority vote is held by the Executive Board whether or not to overturn and reevaluate the decision. If the vote passes, the Executive Board may discuss and make a new ruling by another Simple Majority vote.

% ARTICLE V - VOTING
\article{Voting}
This section outlines the different types of votes and ballots used to make House decisions and defines relevant terminology.
\asection{Definitions}
\asubsection{Total Number of Possible Votes}
The sum of the number of Active members eligible to vote (i.e. not on co-op)
\asubsection{Total Number of Votes Cast}
The Total Number of Votes Cast is defined as the total number of votes received for every voting option minus the number of abstentions.
\asubsection{Quorum}
A Quorum is defined by the minimum number of votes cast required for a vote to be official. It is a fraction or a percentage of the total number of possible votes. Any member present for an Immediate Vote or given a ballot who does not explicitly cast their vote is counted as an Abstention. A Quorum is reached if the Total Number of Votes Cast plus the number of Abstentions is equal to or exceeding the minimum number of votes required.
\asubsection{Proxy Ballot}
A Proxy Vote is defined as any ballot that was cast by one member on behalf of another member. Any member may cast a Proxy Vote for another member who is unable to actually participate in the vote. A Proxy Vote must be explicitly written down and signed by the member not in attendance. The count of all Proxy Vote must be recorded and announced in all votes. Proxy Votes are only permissible where explicitly stated, at the discretion of the person chairing the vote.
\asubsection{Abstention}
An Abstention is defined as a vote indicating a neutral position in the vote. A means to abstain must always be provided in a vote. Abstentions are counted towards a Quorum, but not towards the Total Number of Votes Cast used to determine if a vote passes or not.
\asubsection{Vote Counters}
Vote Counters are the chairperson of the vote, and two randomly selected members.

\asection{Types of Voting}
\asubsection{Balloted Vote}
\asubsubsection{Method of Vote} 
Votes are cast on paper ballots, which provide a means to indicate every possible option in the vote. A ballot is then distributed to each House member eligible to cast a vote and collected in the designated ballot box for a pre-specified length of time. At the end of the voting period, the person chairing the vote collects the ballots, closing the voting period. The Vote Counters then tally the results.
\asubsubsection{Voting Period} 
For constitutional modification, candidate selection, and officer removal votes, the voting period must be at least forty-eight (48) hours in length. For any other type of vote, the voting period must be at least twenty-four (24) hours. The minimum length of the voting period may be explicitly lengthed, but never shortened, in the text describing the actual vote.
\asubsection{Immediate Vote}
\asubsubsection{Method of Vote} 
The person chairing the vote will state all possible ways to vote, then call out each possibility one at a time. The chairing member will count the number of members casting their immediate vote for that possibility.
\asubsubsection{Voting Period} 
An immediate vote lasts as long as it takes for all votes to be tallied.
\asubsection{Secret Immediate Vote}
\asubsubsection{Method of Vote} 
The person chairing the vote will state all possible ways to vote, then call out each possibility one at a time. Votes are kept anonymous. The chairing member and Vote Counters will count the number of members casting their immediate vote for that possibility.
\asubsubsection{Voting Period}
A secret immediate vote lasts as long as it takes for all votes to be tallied.
\asubsection{Batch Vote}
When a Batch Vote is called for by the chair, a subset of the voting docket may be amended to a single vote. A Two-Thirds Immediate Vote is required to allow a Batch Vote to take place. If the call for Batch Vote passes, the subset may then be voted on, otherwise the docket remains unchanged.

\asection{Number of Votes Required}
The Number of Votes Required refers to the numbers required to achieve a quorum and for a vote to pass. Below are listed four standard votes. Numbers for non-standard votes are defined in the section describing the actual vote.
\asubsection{Simple Majority}
In a Simple Majority Vote, a Quorum is reached if the Total Number of Votes Cast is equal to or exceeds one-half the Total Number of Possible Votes. An option in the vote passes if the number of votes cast for that option is larger than the number of votes cast for every other option individually.
\asubsection{Fifty Percent}
In a Fifty Percent Vote, a Quorum is reached if the Total Number of Votes Cast is equal to or exceeds one-half the Total Number of Possible Votes. An option in the vote passes if the number of votes cast for that option exceeds fifty percent of the Total Number of Votes Cast.
\asubsection{Two-Thirds}
In a Two-thirds Vote, a Quorum is reached if the Total Number of Votes Cast is equal to or exceeds two-thirds the Total Number of Possible Votes. An option in the vote passes if the number of votes cast for the option equals or exceeds two-thirds of the Total Number of Votes Cast.
\asubsection{Three Tiered}
When voting on a set of three choices where each choice can be ranked in a definite order, a selection for either the lowest or highest option will only pass if the votes cast for that selection exceed fifty percent of the Total Number of Votes Cast. If neither the highest or lowest selection exceeds fifty percent, the middle option will then automatically be passed. In a Three-Tiered Vote, a Quorum is reached if the Total Number of Votes Cast is equal to or exceeds two-thirds of the Total Number of Possible Votes.

\asection{Ties Between Vote Options}
\asubsection{With Pass/Fail}
If the number of votes cast for the pass option equals the number of votes cast for the fail option, then the vote has failed.
\asubsection{With Multiple Options}
If multiple options may pass, a tie does not present a problem. If only one option may pass, then the vote must be recast or tabled at the discretion of the person chairing the vote. In the event of a tie in an Executive Board vote, the Chairman may cast the tie-breaking vote.

% ARTICLE VI - JUDICIAL
\article{Judicial}
The Executive Board may approve (by simple majority) a request (from a member) for a Judicial proceeding when an official clarification of the Constitution or By-Laws is required, there is a conflict among House Members, or a House interest needs resolution.
\asection{Formation of a Judicial Board}
A Judicial Board is made up of the Chairman, the Evaluations director, and an additional voting member of the Executive Board chosen by a majority vote among the Executive Board. If either the Chairman or Evaluations Director are deemed biased or unfit for a position on the Judicial Board,  they will be replaced by another voting member of the Executive Board by means of a simple majority vote amongst the remaining Executive Board members.
\asection{Judicial Investigation}
The Judicial Board will be responsible for making all necessary inquiries into the matter of the request brought to the Judicial Board.
\asection{Judicial Ruling}
The Judicial Board may present an official ruling following the Judicial Investigation. This ruling may be appealed in the same manner as an Executive Board decision \ref{Appeals}.

\end{document}
